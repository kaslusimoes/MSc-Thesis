% Table of contents formatting
\renewcommand{\contentsname}{Table of Contents}

% Headers and page numbering
\usepackage{fancyhdr}

% Fonts and typesetting
\setmainfont{TeX Gyre Pagella}
\setsansfont{DejaVu Sans}

% Set figure legends and captions to be smaller sized sans serif font
\usepackage[font={footnotesize,sf}]{caption}

\usepackage{siunitx}

% Adjust spacing between lines to 1.5
\usepackage{setspace}
\onehalfspacing
\raggedbottom

% Set margins
\usepackage[top=1.25in,bottom=1.25in]{geometry}

% Chapter styling
\usepackage[grey]{quotchap}
\makeatletter
\renewcommand*{\chapnumfont}{%
  \usefont{T1}{\@defaultcnfont}{b}{n}\fontsize{80}{100}\selectfont% Default: 100/130
  \color{chaptergrey}%
}
\makeatother

% Set colour of links to black so that they don't show up when printed
\usepackage{hyperref}
\hypersetup{colorlinks=true, linkcolor=black}

% Tables
\usepackage{booktabs}
\usepackage{threeparttable}
\usepackage{array}
\newcolumntype{x}[1]{%
>{\centering\arraybackslash}m{#1}}%

% Allow for long captions and float captions on opposite page of figures
% \usepackage[rightFloats, CaptionBefore]{fltpage}

% Don't let floats cross subsections
% \usepackage[section,subsection]{extraplaceins}


% My Math definitions
\DeclareMathOperator*{\argmin}{arg\,min}
\DeclareMathOperator*{\argmax}{arg\,max}
\DeclareMathOperator*{\adj}{adj}
\DeclareMathOperator*{\sign}{sign}

\newcommand{\defeq}{\vcentcolon=}
\newcommand{\mean}[1]{\left\langle #1 \right\rangle}
\newcommand{\Der}[1]{\frac{\mathrm{d}}{\mathrm{d} #1}}
\newcommand{\Derr}[1]{\frac{\mathrm{d^2}}{\mathrm{d^2} #1}}
\newcommand{\der}[2]{\frac{\mathrm{d} #1}{\mathrm{d} #2}}
\newcommand{\derr}[2]{\frac{\mathrm{d^2} #1}{\mathrm{d} #2^2}}
\newcommand{\derrr}[2]{\frac{\mathrm{d^3} #1}{\mathrm{d} #2^3}}
\newcommand{\Del}[1]{\frac{\partial}{\partial #1}}
\newcommand{\Dell}[1]{\frac{\partial^2}{\partial^2 #1}}
\newcommand{\del}[2]{\frac{\partial #1}{\partial #2}}
\newcommand{\dell}[2]{\frac{\partial^2 #1}{\partial #2^2}}
\newcommand{\deldel}[3]{\frac{\partial^2 #1}{\partial #2 \partial #3}}
\newcommand{\funcder}[2]{\frac{\delta #1}{\delta #2}}
\newcommand{\Funcder}[1]{\frac{\delta}{\delta #1}}
\newcommand{\intinf}{\int_{-\infty}^{\infty}}
\newcommand{\intminf}{\int_{\infty}^{-\infty}}
\newcommand{\intxinf}[1]{\int_{#1}^{\infty}}
\newcommand{\intinfx}[1]{\int_{-\infty}^{#1}}

\newcommand{\hamilt}{\mathcal{H}}
\newcommand{\cost}{\mathcal{E}}
\newcommand{\model}{\mathcal{M}}
\newcommand{\prof}{\mathbf{B}}
\newcommand{\profi}[1][i]{B^{#1}}
\newcommand{\student}{\mathbf{J}}
\newcommand{\studenti}[1][i]{J^{#1}}
\newcommand{\measure}[1] {\mathrm{d}#1}
\newcommand{\xx}{\mathbf{x}}

% I get some error here I cannot fix
% \renewcommand{\(}{\left(}
% \renewcommand{\)}{\right)}
% \renewcommand{\[}{\left[}
% \renewcommand{\]}{\right]}
